\documentclass{article}
\usepackage[utf8]{inputenc}
\usepackage{amsmath}
\usepackage{amsfonts}
\usepackage{amssymb}
\usepackage{geometry}
\geometry{a4paper, margin=1in}
\setlength{\parindent}{0pt}
\setlength{\parskip}{1em}

\begin{document}

\section*{Resolução da Prova}

\hrulefill
\subsection*{Questão 1: Estimação por conglomerados}
Uma amostra de 5 fazendas na Bahia (conglomerados) foi selecionada por amostragem aleatória simples sem reposição de uma população de 100 fazendas para estimar a produção média de soja por hectare. Os dados da amostra são:
\begin{itemize}
    \item Fazenda 1: 80 hectares, produção total 400 toneladas
    \item Fazenda 2: 120 hectares, produção total 540 toneladas
    \item Fazenda 3: 100 hectares, produção total 480 toneladas
    \item Fazenda 4: 90 hectares, produção total 450 toneladas
    \item Fazenda 5: 110 hectares, produção total 495 toneladas
\end{itemize}
Estime a produção total de soja por hectare ($\hat{t}$). Forneça uma expressão para a estimativa da variância.

\subsubsection*{Solução}
A questão pede para estimar a "produção total de soja por hectare", o que pode ser interpretado de duas maneiras: a produção \textbf{total} de soja ($\hat{t}_y$) ou a produção \textbf{média} por hectare ($\hat{R}$). Vamos calcular ambos.

\textbf{1. Estimativa do Total de Produção de Soja ($\hat{t}_y$):}
O estimador do total para amostragem de conglomerados em um estágio (com AAS) é dado por:
\[ \hat{t}_y = \frac{N}{n} \sum_{i \in S} y_i \]
Onde $N=100$ (total de fazendas), $n=5$ (fazendas na amostra) e $y_i$ é a produção da fazenda $i$.

Primeiro, somamos a produção das fazendas na amostra:
\[ \sum_{i \in S} y_i = 400 + 540 + 480 + 450 + 495 = 2365 \text{ toneladas} \]

Agora, aplicamos a fórmula do estimador:
\[ \hat{t}_y = \frac{100}{5} \times 2365 = 20 \times 2365 = \mathbf{47300 \textbf{ toneladas}} \]

\textbf{2. Estimativa da Produção Média por Hectare ($\hat{R}$):}
Este é um estimador de razão, onde estimamos a produção total e dividimos pela área total.
\[ \hat{R} = \frac{\sum_{i \in S} y_i}{\sum_{i \in S} M_i} \]
Onde $M_i$ é a área (em hectares) da fazenda $i$.

Somamos a área das fazendas na amostra:
\[ \sum_{i \in S} M_i = 80 + 120 + 100 + 90 + 110 = 500 \text{ hectares} \]

Calculamos a razão:
\[ \hat{R} = \frac{2365}{500} = \mathbf{4.73 \textbf{ toneladas por hectare}} \]

\textbf{Expressão para a Estimativa da Variância do Total ($\hat{V}(\hat{t}_y)$):}
A expressão para a variância estimada do total é:
\[ \hat{V}(\hat{t}_y) = N^2 \left(1 - \frac{n}{N}\right) \frac{s_y^2}{n} \]
Onde $s_y^2$ é a variância amostral dos totais dos conglomerados:
\[ s_y^2 = \frac{1}{n-1} \sum_{i \in S} (y_i - \bar{y}_s)^2 \]
e $\bar{y}_s$ é a média dos totais dos conglomerados na amostra ($\bar{y}_s = 2365/5 = 473$).

\hrulefill
\subsection*{Questão 2: Eficiência na estimação por conglomerados}
No contexto da amostragem por conglomerados, como o coeficiente de correlação intraclasse impacta a eficiência do estimador de Horvitz-Thompson para a média populacional?

\subsubsection*{Solução}
O coeficiente de correlação intraclasse (CCI ou $\rho$) mede o grau de homogeneidade dos elementos dentro dos conglomerados. Seu impacto na eficiência do estimador da média é fundamental:
\begin{itemize}
    \item \textbf{CCI Alto (próximo de +1):} Indica que os elementos dentro de um mesmo conglomerado são muito semelhantes entre si. Isso \textbf{diminui a eficiência} (aumenta a variância) do estimador. A intuição é que, após observar uma unidade no conglomerado, as outras unidades do mesmo conglomerado fornecem pouca informação nova, tornando a amostra menos informativa para um dado número de elementos observados.
    \item \textbf{CCI Baixo (próximo de 0):} Indica que os conglomerados são internamente heterogêneos, ou seja, cada conglomerado tende a ser um "mini-retrato" da população. Isso \textbf{aumenta a eficiência} (diminui a variância) do estimador. Neste cenário ideal, cada conglomerado fornece uma representação fiel da variabilidade populacional.
\end{itemize}
Em resumo, a eficiência do estimador da média em amostragem por conglomerados é \textbf{inversamente proporcional} ao valor do coeficiente de correlação intraclasse. Para maximizar a eficiência, deve-se buscar conglomerados que sejam tão heterogêneos internamente quanto possível.

\hrulefill
\subsection*{Questão 3: Amostragem com probabilidade proporcional ao tamanho (PPT)}
Explique por que a amostragem com probabilidade proporcional ao tamanho (PPT) é frequentemente mais eficiente do que a amostragem aleatória simples (AAS) sem reposição para estimar totais populacionais quando existe uma correlação positiva entre a variável de interesse $(y)$ e uma variável auxiliar de tamanho $x$.

\subsubsection*{Solução}
A maior eficiência da amostragem com PPT em relação à AAS, sob correlação positiva entre $y$ e $x$, deve-se à forma como o estimador de Horvitz-Thompson lida com a variabilidade das unidades.

O estimador de total de Horvitz-Thompson é $\hat{t}_{HT} = \sum_{i \in S} \frac{y_i}{\pi_i}$. A variância deste estimador depende da variabilidade dos valores $\frac{y_i}{\pi_i}$ para as unidades da população.

\begin{enumerate}
    \item \textbf{Na Amostragem Aleatória Simples (AAS):} A probabilidade de inclusão é a mesma para todas as unidades ($\pi_i = n/N$). Se uma unidade com um valor $y_i$ muito grande for selecionada, o termo $\frac{y_i}{n/N}$ será muito grande, introduzindo uma alta variabilidade na estimação. A seleção ou não de uma dessas unidades "gigantes" causa grandes flutuações no valor da estimativa total, resultando em uma variância elevada.

    \item \textbf{Na Amostragem com PPT:} A probabilidade de inclusão $\pi_i$ é feita proporcionalmente ao tamanho $x_i$. Como $y_i$ e $x_i$ têm correlação positiva, unidades com $y_i$ grande também terão $x_i$ grande e, consequentemente, uma $\pi_i$ grande. Isso tem um efeito estabilizador no termo $\frac{y_i}{\pi_i}$. Se a relação for quase linear ($y_i \approx c \cdot x_i$), o termo $\frac{y_i}{\pi_i}$ torna-se aproximadamente constante para todas as unidades, pois a $\pi_i$ "compensa" o tamanho de $y_i$.
\end{enumerate}

\textbf{Conclusão:} Ao tornar os valores ponderados ($\frac{y_i}{\pi_i}$) mais homogêneos entre si, a amostragem PPT reduz drasticamente a variância do estimador do total em comparação com a AAS. Essa redução na variância significa um aumento direto na eficiência.

\hrulefill
\subsection*{Questão 4: Estimação de parâmetros lineares de totais}
Uma pesquisa foi feita para estimar a diferença no número de horas semanais gastas em redes sociais por jovens de dois cursos diferentes na UFBA (A e B). Os resultados das amostragens independentes são:
\begin{itemize}   
\item Curso A: \(\widehat{t}_A = 50.000\) horas; \(\widehat{V}(\widehat{t}_A) = 1.200.000\)
\item Curso B: \(\widehat{t}_B = 42.000\) horas; \(\widehat{V}(\widehat{t}_B) = 900.000\).
\end{itemize}
Estime o contraste entre os totais (\(D = 3 t_A - 2 t_B\)) e a variância dessa estimativa.

\subsubsection*{Solução}
\textbf{1. Estimativa do Contraste ($\hat{D}$):}
O estimador de uma combinação linear de parâmetros é a mesma combinação linear dos estimadores.
\[ \hat{D} = 3 \hat{t}_A - 2 \hat{t}_B \]
Substituindo os valores dados:
\[ \hat{D} = 3 \times (50.000) - 2 \times (42.000) \]
\[ \hat{D} = 150.000 - 84.000 = \mathbf{66.000 \textbf{ horas}} \]

\textbf{2. Estimativa da Variância do Contraste ($\hat{V}(\hat{D})$):}
Para uma combinação linear de estimadores \textbf{independentes}, a variância da combinação é a soma ponderada das variâncias, onde os pesos são os coeficientes ao quadrado.
\[ \hat{V}(\hat{D}) = \hat{V}(3 \hat{t}_A - 2 \hat{t}_B) = 3^2 \hat{V}(\hat{t}_A) + (-2)^2 \hat{V}(\hat{t}_B) \]
Substituindo os valores dados:
\[ \hat{V}(\hat{D}) = 9 \times (1.200.000) + 4 \times (900.000) \]
\[ \hat{V}(\hat{D}) = 10.800.000 + 3.600.000 = \mathbf{14.400.000} \]

\hrulefill
\subsection*{Questão 5: Estimação de parâmetros não-lineares de totais}
Estamos interessados em estimar a variância de um estimador do parâmetro populacional não-linear (produto de totais) $\theta=f(t_y,t_z)=t_y\cdot t_z$, através do produto de dois estimadores de Horvitz-Thompson de totais $\widehat{\theta} = {\widehat{t}_\pi}{_y} \cdot {\widehat{t}_\pi}{_z}$. Descreva passo a passo como poderia aproximar a variância de $\hat{\theta}$.

\subsubsection*{Solução}
Para aproximar a variância de um estimador não-linear como $\hat{\theta} = \hat{t}_y \cdot \hat{t}_z$, utilizamos o \textbf{Método da Linearização de Taylor} (ou Método Delta). O procedimento é o seguinte:

\textbf{Passo 1: Definir a função e o estimador}
A função dos parâmetros é $f(t_y, t_z) = t_y \cdot t_z$. O estimador é $\hat{\theta} = f(\hat{t}_y, \hat{t}_z) = \hat{t}_y \cdot \hat{t}_z$.

\textbf{Passo 2: Linearizar a função}
Aproximamos a função $f$ usando uma expansão de Taylor de primeira ordem em torno dos verdadeiros valores dos totais $(t_y, t_z)$:
\[ \hat{\theta} \approx f(t_y, t_z) + (\hat{t}_y - t_y) \frac{\partial f}{\partial t_y} \bigg|_{(t_y, t_z)} + (\hat{t}_z - t_z) \frac{\partial f}{\partial t_z} \bigg|_{(t_y, t_z)} \]

\textbf{Passo 3: Calcular as derivadas parciais}
Calculamos as derivadas parciais da função $f(t_y, t_z) = t_y \cdot t_z$ e as avaliamos nos pontos $(t_y, t_z)$:
\[ \frac{\partial f}{\partial t_y} = t_z \]
\[ \frac{\partial f}{\partial t_z} = t_y \]

\textbf{Passo 4: Construir a aproximação linear}
Substituímos as derivadas na expansão de Taylor:
\[ \hat{\theta} \approx t_y t_z + (\hat{t}_y - t_y) t_z + (\hat{t}_z - t_z) t_y \]
Isso pode ser reorganizado como:
\[ \hat{\theta} - t_y t_z \approx t_z (\hat{t}_y - t_y) + t_y (\hat{t}_z - t_z) \]

\textbf{Passo 5: Calcular a variância da aproximação}
A variância de $\hat{\theta}$ é aproximada pela variância do lado direito da expressão linearizada. Como $t_y$ e $t_z$ são constantes, temos:
\[ V(\hat{\theta}) \approx V \left( t_z \hat{t}_y + t_y \hat{t}_z \right) \]
Aplicando as propriedades da variância:
\[ V(\hat{\theta}) \approx t_z^2 V(\hat{t}_y) + t_y^2 V(\hat{t}_z) + 2 t_y t_z \text{Cov}(\hat{t}_y, \hat{t}_z) \]

\textbf{Passo 6: Estimar a variância}
Para obter um estimador da variância, substituímos todos os parâmetros populacionais desconhecidos ($t_y$, $t_z$, $V(\hat{t}_y)$, $V(\hat{t}_z)$, $\text{Cov}(\hat{t}_y, \hat{t}_z)$) por seus respectivos estimadores amostrais:
\[ \hat{V}(\hat{\theta}) \approx \hat{t}_z^2 \hat{V}(\hat{t}_y) + \hat{t}_y^2 \hat{V}(\hat{t}_z) + 2 \hat{t}_y \hat{t}_z \widehat{\text{Cov}}(\hat{t}_y, \hat{t}_z) \]
Os termos $\hat{V}(\hat{t}_y)$, $\hat{V}(\hat{t}_z)$ e $\widehat{\text{Cov}}(\hat{t}_y, \hat{t}_z)$ devem ser calculados usando as fórmulas apropriadas para o plano amostral utilizado (por exemplo, as fórmulas de variância e covariância de Horvitz-Thompson ou Yates-Grundy).

\end{document}